%!TEX root = ../main.tex

\section{Photonics}

% need some figures probably

The field of photonics is concerned with the study and manipulation of light.
This endeavor has given rise to countless technologies of great practical and scientific interest.
Most prominently, the use of light as an information carrier has enabled high speed and low loss communications through the use of optical fiber technologies \cite{agrawal_fiber-optic_2012}.
Light is also used extensively for precise detection and measurement in  scientific studies.  
For example, X-ray radiation is now used to observe femtosecond dynamics in chemical reactions \cite{kern_structures_2018}, and laser interferometry was recently used to measure gravitational waves emitted from black hole mergers \cite{ligo_scientific_collaboration_and_virgo_collaboration_observation_2016}.
Apart from these, there are many applications of photonics with significant practical importance ranging from renewable energy \cite{carlson_amorphous_1976,yu_fundamental_2010} to heat transfer \cite{raman_passive_2014,hsu_radiative_2016}.

One of the most important achievements of photonics in the past few decades has been the development of \textit{integrated} photonic devices [integrated].
In this paradigm, rather than constructing devices using macroscopic components, such as lenses and mirrors, they are created on the surface of a chip using techniques common to the semiconductor industry.
Such an approach is appealing as it allows for compact, low cost, and highly functional devices that are also easier to integrate with existing electronic platforms based on composite metal on semiconductor (CMOS) technology [CMOS].
The field of `Silicon photonics' has especially generated much interest in recent years, in which photonic devices integrated on Silicon are employed in applications ranging from optical interconnects for fast data transfer between microchips to large scale integrated photonic circuits [cite Si Pho].

Here, we will primarily explore two emerging technologies based on integrated photonics, (1) Laser-driven particle accelerators on a chip, and (2) optical hardware for machine learning applications.
The approach to laser-driven particle acceleration examined here is referred to as `dielectric laser acceleration', in which charged particles are accelerated by the near field of a patterned dielectric structure driven by an external laser.
As we will show, this technology may benefit greatly from the use of integrated photonic platforms for its eventual practical applications.
Integrated photonics is also a promising candidate for building hardware platforms specialized on machine learning tasks.
As the transmission of an image through an optical lens passively performs a Fourier transform, reconfigurable integrated photonic devices are capable of performing arbitrary linear operations through pure transmission of optical signals through their domain.
As machine learning models are often dominated by linear operations, this technology may provide a platform with higher processing speed, lower energy usage when compared to conventional digital electronics.

\section{Designing of Photonic Devices}

\subsection{Traditional Design Approach}

In any of these applications, the design of the photonic device is of critical importance.
The typical approach to such a process is to use physical intuition to propose an initial structure.
This structure may be parameterized by several \textit{design variables}, such as geometric or material parameters.
These parameters may then optimized, using numerical simulation or experiment, until convergence on a functioning device that further satisfies fabrication constraints, such as minimum feature size, for example.
As an example, if one is interested in designing a device routes input light to different ports for different input wavelengths, one such approach would be to combine several wavelength filters into one device and tune their parameters until the functionality is achieved.
Such an approach, while intuitive, has a number of potential drawbacks.
First, it is dependent on the designer having significant physical intuition about the problem, which is not always available especially in novel applications.
Second, the method of tuning parameters by hand is tedious and the time needed to complete such a task generally scales exponentially with the number of design variables.
This fact means that the designer is practically limited to examining a small number of design variables or only a few select combinations.
The use of few design variables further limits the designer to consider devices within a fixed parameterization.
For example, if one were to designing a device for tailored diffraction or transmission characteristics, he or she may decide to explore grating structures parameterized by tooth height, width, and duty cycle, while ignoring other possible designs.

\subsection{Inverse Design Approach}

\textit{Inverse design} is a radically different approach that has become popularized in photonics within the past decade [inv des].  
In this scheme, the overall performance of the device is defined mathematically through an \textit{objective function}, which is then either maximized or minimized using computational and mathematical optimization techniques.
This approach allows for automated design of photonic devices that are often more compact and higher performance than their traditionally designed alternatives.
Furthermore, this approach allows one to search through a much larger parameter space, typically on the order of thousands to millions of design variables, which allows the design algorithms to often find structures with complexities often extending beyond the intuition of the designer.

The use of inverse design has a long history in other fields, such as mechanics [mech], aerodynamics [aero], and heat transfer [heat].
However, in the past decade, it has been applied successfully to many photonics problems.
A few early examples include the use of inverse design to engineer wavelength splitters [WDM], perfect 90 degree bends in dielectric waveguides [90 Deg], or the design of photonic crystals [PhC].
More recently, it was applied to engineer more exotic phenomena, such as the photonic crystal band structure [PhC band], nonlinear optical responses [nonlin Zin], and metasurfaces [meta].
For a thorough overview of the progress of inverse design in photonics at the time of publishing, we refer the reader to Ref. [A-Rod].

\section{Introduction to Adjoint Method}

As we will explore in detail, the ability to perform inverse design is largely enabled by the ability to efficiently search such a large parameter space.
Typically, this is performed using \textit{gradient-based optimization} techniques, which use local gradient information to iteratively progress through the design space.
In design problems with several degrees of freedom, gradient-based methods typically converge on local minima much faster than more general optimization techniques such as particle swarm optimization or genetic algorithms [cite], which don't typically use local gradient information.

In problems constrained by physics described by linear systems or differential equations, the \textit{adjoint method} is used to compute these gradients.
The adjoint method allows one to compute gradients of the objective function with respect to each of the design parameters in a complexity that is (in practice) independent on the size of the design space.
As such, it is the cornerstone of the inverse design works in photonics and other fields.

Here we give a brief introduction to the mathematics behind the adjoint method.
Many engineering systems can be described by a linear system of equations $A(\bfphi) \bfx = \bfb$, where $A$ is a sparse matrix that depends on a set of parameters describing the system, $\bfphi$.
Solving this equation with source $\bfb$ results in the solution $\bfx$, from which an objective function $J=J(\bfx)$ can be computed. 

The optimization of this system corresponds to maximizing or minimizing $J$ with respect to the set of parameters $\bfphi$.
For this purpose, the adjoint method allows one to calculate the gradient of the objective function $\nabla_\bfphi J$ for an arbitrary number of parameters.
Crucially, this gradient may be obtained with the computational cost of solving only one additional linear system $\hat{A}^T \bar{\bfx} = -\pfrac{J}{\bfx}^T$, which is often called the `adjoint' problem.

As we will show, this method may be readily applied to the inverse design of electromagnetic devices.
In this case, $A$ represents Maxwell's equations describing the device, $\bfx$ are the electromagnetic fields, and $\bfb$ is the electric current source.

\section{Thesis Overview}

Like inverse design, the adjoint method has been known in the applied math community for quite some time, and has been applied to numerous other fields.
Its application to photonics is quite recent, but has had a significant impact.
In this thesis, we will discuss the application of the adjoint method to new applications in photonics.
We will also introduce extensions to the adjoint method, which allow it to be applied to new systems and implemented experimentally.
The thesis is organized as follows
In Chapter 2, we will introduce the mathematical details behind adjoint-based optimization.  
To give a concrete example, we will focus on its application to laser-driven particle accelerators on a chip.
To continue this discussion, in Chapter 3, we will discuss the scaling of laser-driven particle accelerators to longer length scales using photonic integrated circuits.
This discussion will motivate the need to use inverse design for new components, and we will discuss efforts to use such techniques to build these systems experimentally.
In Chapter 4, we will discuss optical hardware platforms for machine learning applications.
The adjoint method will be explored in the context of training an optical neural network, and we will show that its implementation corresponds to the backpropagation algorithm of conventional neural networks.
A novel method for experimentally measuring the gradients obtained through the adjoint method will be introduced in the context of machine learning hardware and we will also discuss our exploration of nonlinear optical activation functions and time-domain machine learning processing using wave physics.
In Chapter 5, we will explore the extension of the adjoint method to new scenarios in photonics, namely nonlinear and periodically modulated systems.
We will conclude in Chapter 6.
