\section{Photonics}

% need some figures probably

The field of photonics is concerned with the study and manipulation of light.
This endeavor has given rise to countless technologies of great practical and scientific interest.
Most prominently, the use of light as an information carrier has enabled high speed and low loss communications through the use of optical fiber technologies [telecomm].
Light is also used extensively for precise detection and measurement in  scientific studies.  
For example, X-ray radiation is now used to observe attosecond dynamics in chemical reactions [chem], and laser interferometry was recently used to measure gravitational waves emitted from black hole mergers [ligo].
Apart from these, there are many applications of photonics with significant practical importance ranging from renewable energy [solar cell] to heat transfer [radiative cooling].

One of the most important achievements of photonics in the past few decades has been the development of \textit{integrated} photonic devices [integrated].
In this paradigm, rather than constructing devices using macroscopic components, such as lenses and mirrors, they are created on the surface of a chip using techniques common to the semiconductor industry.
Such an approach is appealing as it allows for compact, lower cost, and highly functional devices that are also easier to integrate with existing electronic platforms based on composite metal on semiconductor (CMOS) technology [CMOS].
The field of `Silicon photonics' has especially generated much interest in recent years, in which photonic devices integrated on Silicon are employed in applications ranging from optical interconnects for fast data transfer between microchips to large scale integrated photonic circuits.

Here, we will primarily explore two emerging technologies based on integrated photonics, (1) Laser-driven particle accelerators on a chip, and (2) optical hardware for machine learning applications.
The approach to laser-driven particle acceleration examined here is referred to as `dielectric laser acceleration', in which charged particles are accelerated by the near field of a patterned dielectric structure driven by an external laser.
As we will show, this technology may benefit greatly from the use of integrated photonic platforms for its eventual practical applications.
Integrated photonics is also a promising candidate for building hardware platforms specialized on machine learning tasks.
As the transmission of an image through an optical lens passively performs a Fourier transform, reconfigurable integrated photonic devices are capable of performing arbitrary linear operations through pure transmission of optical signals through their domain.
As machine learning models are often dominated by linear operations, this technology may provide a platform with higher processing speed, lower energy usage when compared to conventional digital electronics.

\section{Designing of Photonic Devices}

\subsection{Traditional Design Approach}

In any of these applications, the design of the photonic device is of critical importance.
The typical approach to such a process is to use physical intuition to propose an initial structure.
This structure may be parameterized by several \textit{design variables}, such as geometric or material parameters.
These parameters may then tuned using simulation or experiment until convergence on a functioning device that satisfies some criteria needed for fabrication, such as minimum feature size, for example.
As an example, if one is interested in designing a device routes input light to different ports for different input wavelengths, one such approach would be to combine several wavelength filters into one device and tune their parameters until the functionality is achieved.
Such an approach, while intuitive, has a number of potential drawbacks.
First, it is dependent on the designer having significant physical intuition about the problem, which is not always available especially in novel applications.
Second, the method of tuning parameters by hand is tedious and the time needed to complete such a task scales exponentially with the number of design variables.
This fact means that the designer is practically limited to examining a small number of design variables or only a few select combinations.
The use of few design variables further limits the designer to consider devices within a fixed parameterization.
For example, if one were to designing a device for tailored diffraction or transmission characteristics, he or she may decide to explore grating structures parameterized by tooth height, width, and duty cycle, while ignoring other possible structures.

% describe brute force method for design using all degrees of freedom

\subsection{Inverse Design Approach}

\textit{Inverse design} is a radically different approach that has become popularized in photonics within the past decade [inv des].  
In this scheme, the overall performance of the device is defined mathematically through an \textit{objective function}, which is then either maximized or minimized using computational and mathematical optimization.
This approach allows for automated design of photonic devices that are often more compact and higher performance than their traditionally designed alternatives.
Furthermore, this approach allows one to search through a much larger parameter space, typically on the order of thousands to millions of design variables, which allows the design algorithms to often find structures with complexities often extending beyond the intuition of the designer.

% give brief of inverse design field in photonics

\section{Introduction to Adjoint Method}

As we will explore in detail, the progress of inverse design is largely enabled by the ability to efficiently search such a large parameter space.
This is typically accomplished through the use of the \textit{adjoint method}, which allows one to compute gradients of the objective function with respect to each of the design parameters in a complexity that is practically independent on the size of the design space.
With the gradients, one may then perform gradient-based optimization, such as gradient descent, which typically converges on local minima much faster than global optimization techniques such as particle swarm optimization or genetic algorithms [cite].

% brief explanation, include some mathematics

% history and application of adjoint and inverse design
While inverse design has been applied in numerous other fields, such as aerodynamics, fluid mechanics, and heat transfer, [cite] its application to photonics is quite recent.

\section{Thesis Overview}
% summarize the rest of the sections.

