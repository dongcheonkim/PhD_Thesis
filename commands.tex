
% referencing equations
\newcommand{\eq}[1]{Eq. (\ref{eq:#1})}
% defining equations
\newcommand{\eqdef}[2]{
    \begin{equation}
        #2
        \label{eq:#1}
    \end{equation}
}

% referencing figures
\newcommand{\fig}[1]{Fig. (\ref{fig:#1})}
% defining figures
\newcommand{\figdef}[3]{
        \begin{figure}[htb!] \centering\includegraphics[width=\textwidth]{figures/#1}
        \caption{#3}
        \label{fig:#2}
    \end{figure}
}

% adjoint math
\newcommand{\bfx}{\mathbf{x}}
\newcommand{\bfb}{\mathbf{b}}
\newcommand{\bfphi}{\bm{\phi}}
\newcommand{\invA}{A^{-1}}
\newcommand{\va}{\vec{a}}
\newcommand{\vb}{\vec{b}}
\newcommand{\aj}{\textrm{aj}}

% physical quantities
\newcommand{\vE}{\vec{E}}
\newcommand{\vH}{\vec{H}}
\newcommand{\vJ}{\vec{J}}
\newcommand{\vr}{\vec{r}}
\newcommand{\veta}{\vec{\eta}}
\newcommand{\eps}{\epsilon}
\newcommand{\curl}{\nabla \times}
\newcommand{\dcurl}{\curl \curl}

% more complicated math
\newcommand{\pfrac}[2]{\ensuremath{\frac{\partial #1}{\partial #2}}}
\newcommand{\dfrac}[2]{\ensuremath{\frac{d #1}{d #2}}}
\newcommand{\real}[1]{\ensuremath{ \Re \left\{ #1 \right\} }}
\newcommand{\bracket}[2]{\ensuremath{ \langle #1 , #2 \rangle }}
